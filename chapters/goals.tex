V této kapitole je přiblížena problematika, kterou se tato bakalářská práce zabývá. V závěru jsou zmíněny výhody, které by případné vyřešení problematiky mohlo přinést.

\section{Problematika}
Následující problém si můžeme zobecnint na většinu nástrojů či aplikací, které slouží k administraci nějakého systému. U rozsáhlých systémů platí, že s rostoucím počtem funkcí, které systém nabízí, roste i složitost jeho administrace. Aby administrátor mohl rychle a efektivně spravovat tyto systémy, musí důkladně znát všechny jejich funkce a principy. V případě ZFS se jedná o desítky příkazů, které slouží pro ovládání souborového systému. I proto společnost Oracle vydává mnoho online návodů, které společně s manuálovými stránkami jednotlivých příkazů vytvářejí dobrý zdroj informací pro administrátory ZFS.



Pro zkušené administrátory, kteří se v prostředí tohoto souborového systému pohybují jíž delší dobu, je příkazová řádka pravděpodobně nejrychlejší a nejefektivnější volbou. Potřebný příkaz stačí napsat do příkazové řádky a sytém ho provede. Administrátor může tyto příkazy použít ve skriptech, které může využít pro automatizaci některých činností. Na druhou stranu existují uživatelé, kteří se teprve seznamují se souborovým systémem ZFS a nemají potřebné znalosti. Jelikož v ZFS neexistuje grafické rozhraní, kde by byla funkcionalita sdružena na jednom místě, musí začínající uživatelé často využívat návodů nebo manuálových stránek.
\section{Řešení}
Z výše uvedených důvodů jsem se rozhodl navrhnout a implementovat graficky orientovaný administrátorský nástroj pro ZFS, který by sdružoval funkcionalitu ZFS do jednoho místa. Z tohoto místa by administrátor mohl pohodlně ovládat funkce systému. Tento nástroj by měl být určený především pro seznámení uživatelů se souborovým systémem ZFS a jeho základními funkcemi.

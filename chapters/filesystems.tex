V úvodu této kapitoly je uvedena definice souborového systému, která je následována výčtem nejznámějších zástupců souborových systémů. V závěru této kapitoly porovnáme několik vlastností vybraných souborových systémů.
\section{Souborový systém}
    \label{fs}
    Souborový systém označuje způsob, kterým se organizují data v datových úložištích. Jak už název napovídá, data jsou v úložišti organizována do souborů \cite{fs}.
    Tento systém si o každém souboru udržuje informace, aby bylo později možné zjistit, kde se soubor nachází nebo jak je velký. Dále tento systém umožňuje tyto data číst, měnit nebo vytvářet.

    V dnešní době existuje mnoho způsobů organizace dat, a proto máme k dispozici i mnoho různých souborových systémů. Ty se liší nejenom ve způsobu jak data organizují,
    ale také v možnostech jak velké datové úložiště dokáží adresovat nebo jak velké soubory dokáží spravovat. Další rozlišovacím prvkem může být závislost na operačním systémů.
    Jelikož nejrozšířenější operační systémy jsou systémy typu Unix a Windows, uvedeme několik typických zástupců souborových systémů pro tyto platformy.

    Mezi nejznámější zástupce unixových souborových systémů patří následující systémy.
    \begin{itemize}
      \item UFS
      \item ext2, ext3, ext4
    \end{itemize}

    Naopak mezi nejznámější zástupce souborových systémů pro platformu Windows se řadí následující.
    \begin{itemize}
      \item FAT12, FAT16, FAT32
      \item NTFS
    \end{itemize}

\section{Porovnání}
    V tabulce \ref{fscompare} můžeme vidět porovnání vybraných souborových systémů z hlediska maximální délky názvu souboru, maximální velikosti souboru a maximální velikosti svazku, který dokáží spravovat \cite{data}.
    \begin{table}[]
    \centering
    \caption{Porovnání systémů souborů}
    \label{fscompare}
    \begin{tabular}{|c|c|c|c|}
    \hline
    Název & Max. délka názvu & Max. velikost souboru & Max. velikost svazku \\ \hline
    UFS & 255b & 4 GiB až 256 TiB & 256 TiB \\ \hline
    ext2 & 255b & 16GB až 2TB & 2TB až 32TB \\ \hline
    ext4 & 255b & 16TB & 1EB \\ \hline
    NTFS & 255b & 16TB & 16EB \\ \hline
    ZFS & 255b & 16EB & 16EB \\ \hline
    \end{tabular}
    \end{table} 
V~úvodu této kapitoly je uvedena definice souborového systému, která je následována výčtem nejznámějších zástupců souborových systémů. V~závěru této kapitoly porovnáme několik vlastností vybraných souborových systémů.
\section{Souborový systém}
    \label{fs}
    Souborový systém označuje způsob, kterým se organizují data v~datových úložištích. Jak už název napovídá, data jsou v~úložišti organizována do souborů. Soubory umožňují ukládání informací na disku a adresáře tyto informace umožňují hierarchicky uspořádat \cite{fs}. Tento systém si o~každém souboru udržuje informace, aby bylo později možné zjistit, kde se soubor nachází nebo jak je velký. Dále tento systém umožňuje tyto data číst, měnit nebo vytvářet.

    V~dnešní době existuje mnoho způsobů organizace dat, a proto máme k~dispozici i mnoho různých souborových systémů. Ty se liší nejenom ve způsobu organizace dat,
    ale také ve velikosti datového úložiště, které dokáží adresovat. Dalším rozlišovacím prvkem může být závislost na operačním systému.
    Jelikož nejrozšířenější operační systémy jsou systémy typu UNIX a Windows, uvedeme několik typických zástupců souborových systémů pro tyto platformy.

    Mezi nejznámější zástupce UNIXových souborových systémů patří následující systémy:
    \begin{itemize}
      \item UFS
      \item ext2, ext3, ext4
    \end{itemize}

    Mezi nejznámější zástupce souborových systémů pro platformu Windows se řadí následující systémy:
    \begin{itemize}
      \item FAT12, FAT16, FAT32
      \item NTFS
    \end{itemize}

\section{Porovnání}
    V~tabulce \ref{fscompare} můžeme vidět porovnání vybraných souborových systémů z~hlediska velikosti diskové adresy, velikosti datových bloků a maximální velikosti svazku, který dokáží spravovat.
    \begin{table}
    \centering
    \caption{Porovnání souborových systémů \cite{fs}}
    \label{fscompare}
    \begin{tabular}{|c|c|c|c|}
    \hline
    Název & Velikost adresy & Velikost bloku & Max. velikost svazku \\ \hline
    FAT-32 & 28b & 4KB-32KB & 2TB \\ \hline
    NTFS & 64b & 512B-64KB & 16EB \\ \hline
    ZFS & 128b & 512B-128KB & 256ZB \\ \hline
    \end{tabular}
    \end{table} 
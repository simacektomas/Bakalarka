V této kapitole je popsaný problém, který tato bakalářská práce řeší a přínos případného řešení.

\section{Problematika}
Následující problém si můžeme zobecnint na většinu systému či programů, které slouží k administraci.

Pro efektivní a rychlou správu systémů je nutné danný systém, jeho možnosti a funkce, důkladně znát. Pro rozsáhlé systémy platí, že 
s rostoucím těchto funkcí a možností, které systém nabízí roste i potřeba všechny tyto funkce znát a umět správně ovládat.
V příapdě ZFS, které uživateli nabízí pouze CLI, se jedná o desítky možností a funckcí, kterými může administrátor ZFS ovládat.
Většinu informací o ZFS lze najít v manuálových stránkách nebo na webových stránkách společnosti Oracle, kde se nachází mnoho
podrobných online návodů jak ZFS administrovat. 

Pro zkušené administrátory, kteří se v prostředí tohoto souborového systému
pohybují jíž delší dobu, je CLI pravděpodobně nejrychlejší a nejefektivnější volbou. Na druhou stranu pro většinu začínajících uživatelů,
kteří potřebují nejprve pochopit jak systém funguje, může množství nabízených funkcí přijít zahlcující. 
\section{Řešení}
Z důvodu uvedených výše jsem se rozhodl navrhnout a implementovat graficky orientovaný administrátorský nástroj pro ZFS,
který by sdružoval funkcionalitu ZFS do jedhono místa, odkud by administrátor mohl pohodlně ovládat funkce systému. 
\section{Přínos}

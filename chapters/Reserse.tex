Tato kapitola popisuje strukturu souborového systému Zettabyte, jeho základní funkce a rozebírá existující řešení výše popsaného problému.
\section{Souborový systém Zettabyte}
    \subsection{Souborové systémy}
    Souborový systém je způsob jak organizovat data v datových úložištích. Jak už název napovídá data jsou v úložišti organizována do souborů.
    Tento systém si o každém souboru udržuje informace, aby bylo později možno zjistit, kde se soubor nachází, jak je velký popřípadě jak se jmenuje.
   
    Způsobů jak data organizovat je mnoho, a proto máme k dispozici mnoho různých souborových systémů. Ty se liší nejenom ve způsobu jak soubory organizují,
    ale také jak velké datové úložiště dokáží adresovat nebo jak velké soubory dokáží spravovat. Další rozlišováním může být závislost na operačním systémů.
    
    Mezi nejznámější zástupce unixových systémů souborů patří následující systémy
    \begin{itemize}
      \item UFS
      \item ext2, ext3, ext4     
    \end{itemize}
    
    Naopak mezi nejznámější zástupce platformy Windows se řadí navazující systémy
    \begin{itemize}
      \item FAT12, FAT16, FAT32     
      \item NTFS
    \end{itemize}
    
        \subsubsection{Porovnání}
        Následuje stručné porovnání vybraných souborových systémů z hledska maximální délky názvu souboru, maximální velikosti souboru a maximální velikosti svazku.
        \begin{table}[]
        \centering
        \caption{Porovnání systémů souborů}
        \label{fscompare}
        \begin{tabular}{|l|l|l|l|}
        \hline
        Název & Max. délka názvu & Max. velikost souboru & Max. velikost svazku \\ \hline
        TODO & TODO & TODO & TODO \\ \hline
        \end{tabular}
        \end{table}
    \subsection{Popis}
    Souborový systém Zettabyte
    \subsection{Struktura}
\section{Rozbor existujících řešení} 

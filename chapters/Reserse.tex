Tato kapitola popisuje strukturu souborového systému Zettabyte, jeho základní funkce a rozebírá existující řešení výše popsaného problému.
\section{Souborové systémy}
    Souborový systém je způsob jak organizovat data v datových úložištích. Jak už název napovídá data jsou v úložišti organizována do souborů.
    Tento systém si o každém souboru udržuje informace, aby bylo později možno zjistit, kde se soubor nachází, jak je velký popřípadě jak se jmenuje.

    Způsobů jak data organizovat je mnoho, a proto máme k dispozici mnoho různých souborových systémů. Ty se liší nejenom ve způsobu jak soubory organizují,
    ale také jak velké datové úložiště dokáží adresovat nebo jak velké soubory dokáží spravovat. Další rozlišováním může být závislost na operačním systémů.

    Mezi nejznámější zástupce unixových systémů souborů patří následující systémy
    \begin{itemize}
      \item UFS
      \item ext2, ext3, ext4
    \end{itemize}

    Naopak mezi nejznámější zástupce platformy Windows se řadí navazující systémy
    \begin{itemize}
      \item FAT12, FAT16, FAT32
      \item NTFS
    \end{itemize}

        \subsection{Porovnání}
        Následuje stručné porovnání vybraných souborových systémů z hledska maximální délky názvu souboru, maximální velikosti souboru a maximální velikosti svazku.
        \begin{table}[]
        \centering
        \caption{Porovnání systémů souborů}
        \label{fscompare}
        \begin{tabular}{|l|l|l|l|}
        \hline
        Název & Max. délka názvu & Max. velikost souboru & Max. velikost svazku \\ \hline
        TODO & TODO & TODO & TODO \\ \hline
        \end{tabular}
        \end{table}
\section{Souborový systém Zettabyte}
    %NICM MOC
    Souborový systém Zettabyte byl původně vyvinut společností Sun Microsystems a následně integrován do operačního systémů Solaris od stejnojmenné společnosti.
    Jak už to tak v dnešním světě bývá v roce 2010 společnost Oracle akvizicí společnosti Sun Microsystem získala vlastnicví ZFS i operačního systému Solaris. Veškerý vývoj
    a podpora pro tyto systémy dnes pochází právě od firmy Oracle.
    
    Systém byl původně navrhnut pro použití čistě v operační systému Solaris s čímž se pojily i minimální požadavky pro provoz ZFS. 
    \begin{itemize}
      \item Architektura procesoru SPARC nebo x86 
      \item Operační systém Solaris 10 6/06 nebo novější
      \item Minimální místo na disku 128 MB 
      \item Minimální místo pro vytvoření poolu 64 MB 
      \item Pro optimální výkon ZFS alespoň 1 GB operační paměti
    \end{itemize}
    <CITATE>
    V dnešní době je již bylo ZFS přeneseno i na jiné operačních systém než je Solaris. Jako příklad můžeme vzít operační systém Linux a většinu jeho distribucí <CITATE> 
    
    \subsection{Struktura}
    Většina tradičních souborových systémů se váže na jedno konkrétní zařízení. Aby bylo možné adresovat více zařízení byl zaveden takzvaný volume manager který souborovému
    systému poskytoval iluzi, že se jedná o samostatné zařízení. <CITATE> Ve skutečnosti se pod touto vrstvou mohlo skrývat více disků. ZFS se v tomto směru vydalo opět svojí
    vlastní cestou. Zařízení agreguje do takzvaných poolů, které slouží jako datový základ pro jednotlivé souborové systémy.<IMAGE>
    \subsection{Pool}
    ZFS jako celek se skládá ze dvou hlavních stavebních kamenů. Prvním z nich je takzvaný pool. V terminologii ZFS je to logické sdružení virtuálních zařízeních, které popisuje rozložení a fyzické vlastnosti těchto zařízeních.<CITATE>Pool je tedy jaká si agregace fyzických nebo virtuálních zařízení, která se poskytuje základ pro souborové systémy.

    Souborové systémy v ZFS se už neváží přímo na konkrétní fyzické zařízení jak tomu bylo dříve, ale na pool jako celek. Historické spojení souborového systému s konkrétním fyzickým zařízením přinášela mnohá omezení. Například pokud jsme souborový systém chtěli rozšířit, museli jsme ho celý znovu vytvořit. Tato operace může být časově náročná, ale
    v ZFS tomu tak není.
        \subsubsection{Vytváření poolu}
        Výše popsaný pool můžeme v systému kdykoli dynamicky vytvořit bez potřeby jakéhokoli zásahu do systému. Stačí když máme k dizpozici potřebné zdroje (disk, partition, soubor) k vytvoření námi požadovaného poolu. Pokud máme požadované zdroje, stačí už jen vybrat unikátní jméno pro pool v kontextu ZFS a v systému provést například následující příkaz.
        \begin{verbatim}
        $ zpool create tank c1t2d0 c2t1d0
        \end{verbatim}         
        V případě dostupnosti disků c1t2d0 a c2t1d0, tento příkaz vytvoří pool jménem tank se zdroji  c1t2d0 a c2t1d0.
        \subsubsection{Rozšiřování poolu}
        Výhodou architektury poolů je možnost dynamického přidávání fyzických nebo virtuálních zařížení do poolu a tím získáváme možnost dynamického rozšiřování kapacity celého poolu.
        Všechny souborové systémy vytvořené nad jedním poolem, sdílejí jeho prostředky. Tudíž když dojde k rozšíření kapacity poolu, rozšíříme i potenciální kapacitu souborových systémů unvnitř.

        Kapacita poolu se dá rozšířit hned několika typy zařízení. Mezi nejpoužívanější patří disk a partition nebo RAID. ZFS nabízí možnost do poolu přidat soubor jako zdroj.
        Tato možnost se jeví jako velmi výhodná pro testování nebo zkoušení, jelikož jsem ji sám při vytváření této práce mnohokrát využil. V praxy tato možnost znamená přidání další
        vrstvy ( souborového systému ) mezi ZFS a samotný zdroj dat. Tento fakt přináší značné zpomalení, a proto se tato možnost v praxy téměř nevyužívá.
        
        \subsubsection{RAID} 
        V praxy naopak často používaná možnost jak rozšířit nebo vytvořit pool je pomocí virtuálního zařízení RAID. Tato technologie nám dává možnost rekonstrukce ztracených dat v případě výpadku nějakého disku. Souborový systém nabízí softwarový RAID, který nám umožňuje nad fyzickými zařízeními ( disk, partition, soubor ) vytvářet virtuální zařízení následujících typů.
        \begin{itemize}
          \item RAID1 - zrcadlení
          \item RAIDZ1 - stripování s jendím paritním diskem
          \item RAIDZ2 - stripování se dvěma paritními disky
          \item RAIDZ3 - stripování se třemi paritními disky
        \end{itemize} 
        <IMAGE>
        
        Zrcadlení (RAID1) můžeme pomocí ZFS vytvořit nad dvěma a více zařízeními. Tento typ virtuálního zařízení pak jednoduše replikuje všechny data z disků na všechny ostatní disky
        ve virtuálním zřízení. Pokud zrcadlení vytvoříme nad třemi disky, nakonec budou data na všech třech discích stejná a v případě výpadku jednoho z nich o data nepřijdeme. Jiná situace je samozřejmě v případě výpadku všech třech disků najednou. RAID nám v tomto případě nepomůže a všechny data ztratíme. Nicméně pravděpodobnost, že dojde k výpadku všech
        tří disků najednou, než stihneme alespoň jeden vyměnit, je malá.
        \begin{lstlisting}[language=bash,caption={Vytváření mirror}]
        $ zpool TODO mirror c1t1d0 c1t2d0 c2t1d0
        \end{lstlisting}   
        
        Stripování (RAIDZN) je technika ukládání dat na disk v takzaných stripech. Vezměme v úvahu například virtuální zařízení RAIDZ1 s jedním paritním diskem, do kterého přidáme tři disky. Pokud pokud zapisujeme na toto virtuální zařízení data jsou tyto data rozdělena do stripů. Stripe je základní jednotou zápisu dat a v našem případě se část stripu ukládá na první disk, část na druhý a poslední část na třetí disk. Z dat je vypočtena takzvaná parita, která se následně uloží na paritní disk. Z této hodnoty v případě výpadku jednoho disku dokážeme zpět dopočítat data, která byla uložena na nefunkčním disku. V případě výpadku dvou a více disků najednou o data opět přijdeme. V případě stripování nám RAIDZ přináší i výhodu ve čtení dat z disků. Vzhledem k tomu, že data jsou rozložena na více discích, dokážeme číst z více disků najednou a tím docílit vyšší rychlosti čtení.
        \begin{lstlisting}[language=bash,caption={Vytváření raidz1}]
        $ zpool create raidz1 c1t1d0 c1t2d0 c2t1d0 c2t2d0
        \end{lstlisting}   
        
        V případě stripování s více paritními disky jsme schopni obnovit data při výpadku dvou disků při dvou paritních discích atp.. Výhoda vyšší čtecí rychlosti je stále zachována, protože můžeme data načítat z více disků najednou.
        \subsection{Zrušení poolu}
        Stejně jako můžeme dynamicky pool vytvářet a rozšiřovat, můžeme pool i kdykoli zničit a uvolnit tak prostředky, které pool využíval. Tyto prostředky jsou ihned k dispozici a můžeme z nich vytvořit nový pool nebo je použít rozšíření jiného poolu. Jelikož všechny souborové systémy vytvořené nad poolem sdílejí jeho prostředky, jsou tyto souborové systémy zničeny spolu s poolem. Ke zničení poolu, který jsme dříve vytvořili můžeme použít následující příkaz.
        \begin{verbatim}
        $ zpool destroy tank
        \end{verbatim}   
        Pokud v systému existuje pool jménem tank, příkaz ho zničí bez ohledu na to kolik souborových systému nad tímto poolem bylo vytvořeno.         
        \subsection{Vlastnosti poolu}
        ZFS si o každém vytvořeném poolu udržuje infromace, které používá pro správu. Každý pool má tedy seznam vlastnotí, které tento pool popisují a jednoznačně určují.
        Hlavním identifikátorem poolu je jeho jméno. Toto jméno musí být v kontextu celého ZFS unikátní, protože je používáno v příkazech manipulujích s danným poolem.
        
        Vlastnosti můžeme rozdělit na dvě skupiny. Na jedné straně jsou statické informace, které popisují stav poolu a nedají se změnit přímo. Jsou takzvaně read-only. Pod tímto druhem informací si můžeme představit informace týkající se úložného prostoru jako například velikost, obsazenost a volné místo. Velikost poolu nemůžeme změnit přímo nastavením vlastnosti na jinou hodnotu, ale můžeme přidat do poolu další zdroj, který tuto informaci změní. 
        
        Na druhé straně jsou informace, které můžeme měnit přímo buďto dynamicky, když je pool vytvořen nebo když pool vytváříme nebo importujeme. Nastavit u poolu můžeme například vlastnot readonly, která zajistí, že od doby kdy byla tato vlasnot nastavena lze z daného poolu data pouze číst a nikoliv data zapisovat. Další zajímavou vlastností je vlastnot bootfs, která odkazuje na souborový systém, který slouží pro boot operačního systému. Manuální nastavovaní této vlastnosti se nedoporučuje, protože špatné nastavení může vést k nenačtení operačního systému.
        
        Pro stručnost jsem z následujícího příkazu, který administrátorovi zobrazuje dostupné infromace o poolu, vybral pouze nejnutnější vlastnoti.
         \begin{verbatim}
        $ zpool get all rpool
        NAME   PROPERTY       VALUE                SOURCE
        rpool  allocated      11.9G                -              
        rpool  bootfs         rpool/ROOT/solaris   local        
        rpool  capacity       38%                  -        
        rpool  dedupratio     1.19x                -
        rpool  delegation     on                   default
        rpool  failmode       wait                 default
        rpool  free           18.9G                -
        rpool  guid           9953515090955426044  -
        rpool  health         ONLINE               -        
        rpool  listsnapshots  on                   local
        rpool  readonly       off                  -
        rpool  size           30.8G                -
        rpool  version        37                   default
        \end{verbatim} 
        
        
    \subsection{Souborový systém}
    Druhým ze stavebních kamenů ZFS jsou samotné souborové systémy. Jak jsem již zmínil dříve, souborový systém je způsob organizace dat do souborů v datových úložištích.
    Tradiční souborové systémy využívají k organizaci různých datových struktur. Souborový UFS využívá takzvaných inodů, což je datová struktura držící attributy souboru a informace o datových blocích. Souborový systém FAT nadruhou stranu využívá takzvané file allocation table. ZFS k tomuto účelu využívá stromové datové struktury, kterou můžete vidět na <IMAGE><CITATE Struktura>. Tato stromová struktura přináší zajímavý výhody o kterých si povíme v další kapitolách.
    \subsection{Hierarchie souborových systémů}
    Jednou z vlastností ZFS je možnost hieararchického vnořování souborových systémů do sebe. Fakt, že samotný pool jakožto zdroj místa pro souborové systémy se chová jako samostatný souborový systém, tuto vlastnost dokazuje. Přesvědčit se o tom můžeme pomocí následujícího příkazu, kterému místo jéma souborového systému předáme jméno poolu. Pokud příkaz proběhne dobře vypíše základní informace o souborovém systému.
    \begin{verbatim}
    $ zfs list rpool
    NAME    USED  AVAIL  REFER  MOUNTPOINT
    rpool  12.1G  18.2G  4.85M  /rpool
    \end{verbatim}
    Pokud by takovýto souborový systém nexistoval, příkaz by vrátil chybovou hlášku.
    \subsection{Vytváření souborového systému}        
    
    Vytváření souborových systémů v ZFS je velice jednoduché. Jediné co potřebujeme k vytovření souborového systému je jednoznačný identifikátor. Tento identifikátor se skládá ze jména poolu, ve kterém chceme systém vytvořit a jména souborového systému. ZFS Nám dovoluje vytvořit v jednom poolu více souborových systémů, které prostředky sdílí rodičovského poolu.    
    \begin{verbatim}
    $ zfs create rpool/test  
    \end{verbatim} 
    Po provedení příkazu ZFS automaticky přiřadí nový souborový systém do požadovaného poolu. Navíc v jednom poolu můžeme takto vytvořit více souborových systémů, které do sebe můžeme dále vnořovat. O hierarchi více v kapitole \ref{hiararchy}.
    \subsection{Zrušení souborového systému}
    Rušení souborový systému je v ZFS stejně snadné jako jejich vytváření.  
    \subsection{Vlastnosti souborového systému}    
    \label{hiararchy}
    \subsection{Konzistence}
    Starší souborové systémy jako například UFS zapisují data přímo do bloku, který mění. To může v některých případech, jako je selhání systému nebo výpadek proudu, vést k nekonzistenci souborového systémů. V takovém případě je nutné celý souborový systém zkontrolovat napříkald pomocí příkazu <code>fsck. Bohužel ani ten v některých případech
    nedokáže některé nekonzistence opravit. <CITATE>

    Některé systémy souborů využívají žurnálů k udržení konzistence. Žurnál jse speciální záznam kam se ukládá co a kde se bude měnit, poté
    je provedena vlastní akce. A v poslední fázi se do žurnálu zapíše, že akce byla provedena. Když systém zkolabuje v jakémkoli kroku, je možné ho při startu například pomocí
    fsck dostat opět do konzistentního stavu.

    ZFS přichází s transakčním systémem a technikou copy on write. Tato technika zajišťuje neustálou konzistenci souborového systému i v okamžiku výpadku, a proto není třeba žurnálu. Data na disku nejsou nikdy přímo přepisována.

    Transakce může vypadat následovně. Nejprve dojde ke zkopírování bloků, které mají být změněny. Poté dojde kopii a změně metadat, které se týkají změněných bloků. V poslední
    řadě dojde k atomické operaci, která připojí nově vzniklou větev se novými resp. změněnými bloky do stromu souborového systémů. Pokud v průběhu transakce dojde k výpadku
    celý souborový systém zůstává konzistentní, protože nebyla provedena atomická operace připojení k hlavnímu stromu.
    \begin{figure}[h]
        \caption{Porovnání systémů souborů}
        \label{cow}
        \includegraphics{cow.pdf}
    \end{figure}

    \subsection{Kontrolní součty}
    \subsection{Deduplikace}
    \subsection{Snapshot}
\section{Rozbor existujících řešení}


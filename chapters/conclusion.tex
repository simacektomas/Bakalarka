Cílem této bakalářské práce bylo vytvořit graficky orientovaný administrátorský nástroj sloužící pro správu souborového systému Zettabyte. Tento cíl byl úspěšně naplněn.

Při vývoji aplikace byl kladen důraz především na bezpečnost, jednoduchost používání a budoucí rozšiřitelnost. Bezpečnost přenosu dat mezi klientem a webovým serverem je zajišťována pomocí HTTPS protokolu, který data šifruje. Do aplikace je umožněn vstup pouze autentizovaným uživatelům, jelikož nástroj umožňuje provádět nebezpečné příkazy. Autentizace uživatelů se provádí oproti lokální databázi uložené v souboru.

Celá aplikace byla integrována do operačního systému Solaris, který využívá souborový systém ZFS v základu. Integrace do systému umožňuje administrátorovi aplikaci jednoduše ovládat a v případě nějaké chyby mu poskytne informace potřebné k jejímu odstranění. Některé funkce a vlastnosti aplikace se dají nastavovat pomocí konfiguračních souborů. Tato možnost administrátorovi umožňuje aplikace přizpůsobit jeho požadavkům. 

Výsledná implementace administrátorkého nástroje neobsahuje všechny funkce správy souborového systému ZFS, a proto bylo nutné navrhnout jeho strukturu tak, aby se dala v budoucnu jednoduše rozšířit o nové funkce. K tomuto účelu bylo využito objektové architektury MVC, která odděluje jednotlivé vrstvy aplikace tak, že jsou na sobě minimálně závislé. Pro rozšíření funkcionality aplikace stačí rozšířit nebo modifikovat jen některé z těchto vrstev. Část aplikace, která zajišťuje komunikaci se souborovým systémem ZFS, byla rozdělena do modulů. Jednotlivé moduly zajišťují 

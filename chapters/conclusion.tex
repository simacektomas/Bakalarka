Souborový systém ZettaByte je jeden z nejpokročilejších souborových systémů, se kterými se dnes můžeme setkat. Ke správě dat na disku využívá moderních principů, které mimo jiné zajišťují úsporu místa, neustálou konzistenci dat nebo elegantní tvorbu snapshotů. Nevýhodou tohoto systému je absence přehledného grafického rozhraní, které by umožňovalo jeho správu. Z tohoto důvodu bylo hlavním cílem této bakalářské práce vytvořit graficky orientovaný administrátorský nástroj sloužící pro správu souborového systému Zettabyte. Tento cíl byl úspěšně naplněn.

Vytvořená aplikace umožňuje administrátorovi jednoduše a komfortně spravovat souborový systém ZFS. Poskytuje základní funkce pro administraci ZFS poolů a souborových systémů. Dále aplikace umožňuje přehledné zobrazování aktuálního stavu ZFS nebo dynamické zobrazení procentuálního využití procesoru. V~neposlední řadě nástroj umožňuje sledovat historii jednotlivých příkazů, které byly v~ZFS provedeny. Funkcionalita aplikace je postavena především nad ZFS příkazy \emph{zfs} a \emph{zpool} a nad systémovými příkazy \emph{fdisk},\emph{format} a \emph{sar}.

Nástroj je implementován pomocí webového rozhraní, které se uživateli prezentuje pomocí jednotlivých HTML stránek. Uživatel se může po těchto stránkách libovolně pohybovat bez nutnosti hlubší znalostí architektury aplikace. Stránky jsou přenášeny prostřednictvím počítačové sítě mezi webovým serverem a klientským prohlížečem. Z tohoto důvodu je k aplikaci možné přistupovat ze vzdáleného počítače.

Při vývoji aplikace byl kladen důraz především na bezpečnost, jednoduchost používání a budoucí rozšiřitelnost. Bezpečnost přenosu dat mezi klientem a webovým serverem je zajišťována pomocí HTTPS protokolu, který data šifruje. Do aplikace je umožněn vstup pouze autentizovaným uživatelům, jelikož nástroj umožňuje provádět nebezpečné příkazy. Autentizace uživatelů se provádí oproti lokální databázi uložené v~souboru. Nástroj má dva režimy bezpečnosti, které povolují resp. zakazují práci se systémovým poolem.

Celá aplikace byla integrována do operačního systému Solaris, pro který byl souborový systém ZFS původně vyvíjen. Integrace do systému umožňuje administrátorovi aplikaci jednoduše ovládat a v~případě nějaké chyby mu poskytne informace potřebné k~jejímu odstranění. Jednotlivé vlastnosti a funkce aplikace se dají nastavit pomocí konfiguračních souborů. Tato možnost administrátorovi umožňuje aplikaci přizpůsobit jeho požadavkům. Dále byla v~operačním systému Solaris vytvořena speciální role, pod kterou se výsledný nástroj spouští. Tento krok aplikaci zajišťuje práva k~vykonávání potřebných příkazů a minimalizuje bezpečnostní rizika.

Z časových důvodů nebylo možné do výsledné aplikace zahrnout všechny funkce správy souborového systému ZFS, a proto byla její struktura navržena tak, aby se dala v~budoucnu jednoduše rozšířit o~nové funkce. K~tomuto účelu bylo využito objektové architektury MVC, která odděluje jednotlivé vrstvy aplikace tak, že jsou na sobě minimálně závislé. Pro rozšíření funkcionality nástroje stačí rozšířit nebo modifikovat jen některé z~těchto vrstev. Vrstva zajišťující komunikaci se souborovým systémem ZFS byla dále rozdělena do modulů. Jednotlivé moduly se specializují na určité oblasti správy ZFS a zajišťují tak aplikaci potřebnou funkcionalitu. Aplikaci je možné o tyto moduly kdykoli rozšířit.

Nejbližší rozšíření aplikace by mělo spočívat v implementaci modulů, které doplní funkcionalitu aplikace o chybějící funkce. Vzhledem ke struktuře aplikace by to neměl být větší problém. Výsledná implementace aplikace umožňuje správu jednoho souborového serveru. Další vylepšení aplikace by mohlo spočívat ve správě více těchto serverů. Bylo by nutné centralizovat sběr dat a na jednotlivé souborové servery umístit sondy, které by zajišťovaly sběr dat.

Pro usnadnění instalace byl vytvořen balíček \emph{wzfsadm-i386.pkg}, který obsahuje všechny zdrojové kódy aplikace a zajišťuje celkovou instalaci nástroje se všemi potřebnými komponenty.

Všechny cíle bakalářské práce a požadavky stanovené na aplikaci se podařilo splnit. Výsledná implementace administrátorského nástroje sloužícího pro správu souborového systému ZFS je dostupná na přiloženém médiu.





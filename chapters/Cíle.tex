V této kapitole je přiblížena problematika, kterou se tato bakalářská práce zabývá. V závěru jsou zmíněny výhody, které by případné vyřešení problemantiky mohlo přinést.

\section{Problematika}
Následující problém si můžeme zobecnint na většinu nástrojů či aplikací, které slouží k administraci nějakého systému.
% CHTELO BY TO PORADNE DOSPECIFIKOVAT
Aby administrátor mohl rychle a efektivně spravovat tento systém, musí důkladně znát všechny jeho funkce a možnosti. U rozsáhlých sytémů platí, že s rostoucím počtem funkcí , které systém nabízí, roste i složitost jeho administrace. V příapdě ZFS, které uživateli nabízí pouze rozhraní na příkazové řádce, se jedná o desítky příkazů, kterými může administrátor ZFS ovládat. I přes to, že je veškerá funkcionalita ZFS popsána v manuálnových stránkách, je nutné tyto příkazy znát. 
Většinu informací o ZFS lze najít v manuálových stránkách nebo na webových stránkách společnosti Oracle, kde se nachází mnoho
podrobných online návodů jak ZFS administrovat.

Pro zkušené administrátory, kteří se v prostředí tohoto souborového systému pohybují jíž delší dobu, je příkazová řádka pravděpodobně nejrychlejší a nejefektivnější volbou. Potřebný příkaz stačí napsat do příkazové řádky a sytém ho provede. Na druhou stranu existují uživatelé, kteří se teprve seznamují se souborvým systémem ZFS a nemají potřebné znalosti. Jelikož v ZFS nexistuje grafické rozhraní, kde by byla funkcionalita sdružena na jednom místě, musí začínající uživatelé využívat stále manuálových stránek, aby se dozvěděli k čemu danný příkaz slouží. K tomuto účelu společnost Oracle, která je nyní vlastníkem souborového systému ZFS, vydává mnoho online návodů, jak tento systém administrovat. 
\section{Řešení}
Z výše uvedených důvodů jsem se rozhodl navrhnout a implementovat graficky orientovaný administrátorský nástroj pro ZFS,
který by sdružoval funkcionalitu ZFS do jedhono místa, odkud by administrátor mohl pohodlně ovládat funkce systému. Tento nástroj by měl být určený především seznámení se uživatelů se souborovým systémem ZFS a jeho funkcemi.

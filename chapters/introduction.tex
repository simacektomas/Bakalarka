Tato práce se věnuje administraci souborového sytému Zettabyte. Popisuje návrh a implementaci grafického nástroje pro jeho monitorování a správu.

Souborový systém Zettabyte (ZFS) je jeden z~mnoha souborových systémů, se kterými se dnes můžeme setkat.
Byl vyvinut společností Sun Microsystems a integrován do operačního systému Solaris. Tento souborový systém dokáže spravovat velké množství dat díky své 128-bitové architektuře a mimo jiné nabízí funkce pro ověřování integrity dat, správu fyzických úložišť, vlastní softwarový RAID a v~neposlední
řadě také šifrování dat. Jedním z~nedostatků tohoto souborového systému je absence grafického rozhraní pro jeho správu. ZFS nabízí administrátorovi pouze rozhraní na příkazové řádce.

Hlavním cílem této bakalářské práce je navrhnout a implementovat graficky orientovanou nadstavbu souborového systému ZFS, která bude umožňovat jeho monitorování a správu. Tento nástroj bude koncentrovat funkcionalitu ZFS do jednoho místa, aby usnadnil uživateli správu tohoto souborového sytému.

Struktura této bakalářské práce se skládá ze šesti hlavních kapitol.
V~první kapitole je objasněn důvod, proč by měl být tento administrátorský nástroj vytvořen a jaký přínos by jeho implementace měla pro různé skupiny uživatelů souborového systému ZFS.
Druhá kapitola popisuje účel souborových systémů a představuje jejich nejznámější zástupce. V~závěrečné části této kapitoly je uvedeno krátké porovnání základních parametrů vybraných souborových systémů.
V~úvodu třetí kapitoly je stručně představen souborový systém ZFS, jeho historie a vnitřní struktury. Hlavním obsahem této kapitoly je představení základních administračních funkcí a zajímavých principů, které může administrátor ZFS využívat.
Čtvrtá kapitola rozebírá několik vybraných nástrojů, které slouží pro monitorování nebo administraci ZFS. Hlavním účelem analýzy těchto nástrojů je objevení jejich kladných a záporných stránek. Implementace výsledného nástroje by se mohla opírat o~tyto poznatky a naopak by se mohla vyvarovat případným chybám a problémům, které vybrané nástroje obsahují.
Úvod páté kapitoly definuje požadavky, které bude implementovaný nástroj splňovat. Hlavní obsah této kapitoly se týká návrhu architektury, volby uživatelského rozhraní a analýzy bezpečnostních rizik administrátorského nástroje.
Téma poslední šesté kapitoly je samotná implementace navrhnutého nástroje. V~této kapitole jsou představeny a stručně popsány jednotlivé komponenty aplikace. Pozornost je také věnována integraci nástroje do operačního systému Solaris a implementaci bezpečnostních opatření, které budou sloužit pro bezpečný chod aplikace.
Po hlavních kapitolách následuje stručný závěr, kde je shrnuto, zda bylo dosaženo stanovených cílů a také jsou zde uvedeny některé možnosti budoucího rozšíření implementovaného nástroje.

Nezbytnou součástí této práce jsou zdrojové kódy aplikace a uživatelská příručka, která popisuje instalaci, konfiguraci a spuštění nástroje. Aplikace i příručka jsou dostupné na přiloženém médiu.




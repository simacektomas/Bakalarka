V této kapitole je uvedena definice souborového systému a také jsou zde představeni nejznámější zástupci. V závěru této kapitoly vybrané zástupce porovnáme v několika ohledech.
\section{Souborový systém}
    \label{fs}
    Souborový systém označuje způsob, kterým se organizují data v datových úložištích. Jak už název napovídá, data jsou v úložišti organizována do souborů. <CITATE Soubor>
    Tento systém si o každém souboru udržuje informace, aby bylo později možné zjistit, kde se soubor nachází, jak je velký popřípadě jak se jmenuje. Dále tento systém umožnuje tyto data číst, měnit nebo vytvářet.

    V dnešní době existuje mnoho způsobů organizace dat, a proto máme k dispozici i mnoho různých souborových systémů. Ty se liší nejenom ve způsobu jak data organizují,
    ale také v možnostech jak velké datové úložiště dokáží adresovat nebo jak velké soubory dokáží spravovat. Další rozlišovacím prvkem může být závislost na operačním systémů.
    Jelikož nejrozšířenější operační systémy jsou systémy typu Unix a Windows, uvedeme několik typických zástupců souborových systémů pro tyto platformy.

    Mezi nejznámější zástupce unixových souborových systémů patří následující systémy.
    \begin{itemize}
      \item UFS
      \item ext2, ext3, ext4
    \end{itemize}

    Naopak mezi nejznámější zástupce pro platformu Windows se řadí navazující systémy.
    \begin{itemize}
      \item FAT12, FAT16, FAT32
      \item NTFS
    \end{itemize}

\section{Porovnání}
    V tabulce \ref{fscompare} můžeme vidět porovnání vybraných souborových systémů z hledska maximální délky názvu souboru, maximální velikosti souboru a maximální velikosti svazku, který dokáží spravovat.
    \begin{table}[]
    \centering
    \caption{Porovnání systémů souborů}
    \label{fscompare}
    \begin{tabular}{|l|l|l|l|}
    \hline
    Název & Max. délka názvu & Max. velikost souboru & Max. velikost svazku \\ \hline
    TODO & TODO & TODO & TODO \\ \hline
    \end{tabular}
    \end{table} 
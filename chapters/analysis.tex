V následující kapitole je rozebráno několik nástrojů sloužících pro monitorování a správu souborového systému ZFS. Hlavním předmětem analýzy jsou výhody a nevýhody implementovaných řešení. Výhody těchto nástrojů by mohly pomoci ke správnému návrhu vlastního řešení administračního nástroje pro ZFS.
\section{zfsmon}
V analýze implementovaných řešení je kladen důraz především na nástroj \emph{zfsmon}. Tento nástroj slouží pro monitorování souborových systémů ZFS na více oddělených souborových serverech. Aplikace se dá rozdělit na dvě hlavní části, které spolu spolupracují.

První částí je skript, který běží na počítači se souborovým systémem ZFS. Tento skript je spouštěn pomocí \emph{crontabu} a zajišťuje aktualizaci informací v databázi, která je součástí webové aplikace. V UNIXových operačních systémech \emph{crontab} zajišťuje automatické spouštění programů v určitých intervalech. Doporučený interval pro spouštění skriptu nástroje \emph{zfsmon} je 15 minut \cite{zfsmon}.

Druhou částí aplikace \emph{zfsmon} je samotná webová aplikace, která se stará o zobrazování dat z databáze. V databázi se nacházejí informace o stavu jednotlivých souborových systémů z různých serverů. Aplikace zobrazuje tyto informace klientovi v podobě HTML stránek.

Velkou výhodou této aplikace je, že dokáže zobrazovat informace z více oddělených úložišť najednou. Na druhou stranu způsob zpracování dat přináší značnou nevýhodu, která spočívá v intervalu aktualizace dat v databázi. V průběhu tohoto intervalu aplikace znemožňuje uživateli vidět aktuální data ze souborového serveru. Uživatel je nucen počkat na konec tohoto intervalu, kdy jsou opět nahrána aktuální data do databáze. Další nevýhodou tohoto nástroje je absence veškeré funkcionality, která by se týkala administrace souborového systému ZFS.

Tento nástroj nám tedy umožňuje kvalitně monitorovat informace z více zdrojů najednou, ale neposkytuje nám možnost administrace monitorovaných souborových systémů.
\section{zfswatcher}
Druhým nástrojem, který slouží pro monitorování souborového systému ZFS, je nástroj jménem \emph{zfswatcher}.

Tento nástroj slouží jako webové rozhraní pro sledování souborového systému ZFS. Stejně jako předchozí nástroj zpracovává periodicky data a zobrazuje je pomocí HTML stránek, s čímž souvisí stejný problém, který se vyskytuje u předchozího nástroje. Oproti nástroji \emph{zfsmon} umí zobrazovat informace pouze z jednoho souborového serveru. Umí však poslat upozornění v podobě e-mailu nebo zalogování do logu vždy, když dojde ke změně stavu ZFS \cite{zfswatcher}.
\section{ZFS administration console}
Na závěr této kapitoly zmíníme ještě nástroj \emph{smcwebserver}. Tento nástroj je součástí operačního systému Solaris 10 6/06 a umožňuje administrátorovi spravovat některé funkce tohoto operačního systému. Mimo jiné se část tohoto nástroje věnuje právě správě souborového systému ZFS \cite{smc}. Bohužel v novější verzi Solarisu 11 se tento nástroj již nevyskytuje. Jako jediný ze zmíněných nástrojů právě \emph{smcwebserver} nabízí administrátorovi funkce pro správu souborového systému ZFS. 

Tato práce se věnuje návrhu a implementaci grafického nástroje pro monitorování a správu souborového systému Zettabyte.

Souborový systém Zettabyte, dále jen ZFS ( Zettabyte file system ), je jeden z mnoha souborových systémů, se kterými se dnes můžeme setkat.
Byl vyvinut společností Sun Microsystems a integrován do operačního systému Solaris. Tento souborový systém dokáže spravovat velké množství dat
díky své 128-bitové architektuře a mimo jiné nabízí funkce pro ověřování integrity dat, správu fyzických úložišť jako například vlastní 
softwarový RAID a v neposlední řadě také šifrování dat.

Hlavním cílém této bakalářské práce je navrhnout a implementovat graficky orientovanou nadstavbu souborového systému Zettabyte, která bude zjednodušovat
a zpříjemňovat správu tohoto systému.

Struktura této bakalářské práce se skládá ze čtyř hlavních kapitol. V první kapitole je vysvětluji, proč by měl být tento administrátorský nástroj
implementován a jaký přínos by měl pro různé skupiny uživatelů. V druhé kapitole se věnuji analýze struktury ZFS a jeho funkcí. Dále je v této kapitole
proveden průzkum již vytvořených grafických nástrojů pro správu ZFS. V předposlední kapitole jsou stanoveny požadavky a popsán návrh výsledného nástroje.
Ve čtvrté a zároveň poslední kapitole je rozebrána implementace aplikace, její instalace, nastavení a samotné spuštění. Po kapitolách následuje sturčný 
závěr, kde shrnuji zda bylo dostaženo stanovený cílů a uvadím možnosti rozšíření výsledné aplikace.

Na přiloženém médiu jsou dostupné zdrojové kódy administrátorského nástroje.





Tato práce se věnuje návrhu a implementaci grafického nástroje pro monitorování a správu souborového systému Zettabyte.

Souborový systém Zettabyte, dále jen ZFS (Zettabyte file system), je jeden z mnoha souborových systémů, se kterými se dnes můžeme setkat.
Byl vyvinut společností Sun Microsystems a integrován do operačního systému Solaris. Tento souborový systém dokáže spravovat velké množství dat
díky své 128-bitové architektuře a mimo jiné nabízí funkce pro ověřování integrity dat, správu fyzických úložišť, vlastní softwarový RAID a v neposlední
řadě také šifrování dat. Hlavním nedostatkem ZFS je absence pohodlného grafického rozhraní pro jeho správu.

Hlavním cílém této bakalářské práce je navrhnout a implementovat graficky orientovanou nadstavbu souborového systému Zettabyte, která bude zjednodušovat
a zpříjemňovat správu tohoto systému.

Struktura této bakalářské práce se skládá ze TODO hlavních kapitol.
V první kapitole je vysvětleno, proč by měl být tento administrátorský nástroj
implementován a jaký přínos by měl pro různé skupiny uživatelů.
Druhá kapitola popisuje souborové systémy obecně, představuje jejich nejznámější zástupce a v závěrečné části této kapitoly uvádím krátké porovnání vybraných souborových systémů.
Ve třetí kapitole jsou přiblíženy záklaní principy souborového systému ZFS. Popisuji zde některé funkce, které ZFS administrátorovi nabízí.
Čtvrtá kapitola se zabývá analýzou implementovaných grafických rozhraních, kde se snažím najít výhody těchto nástrojů, které bych mohl použít pro následnou implementaci. Na druhou stranu se snažím najít i problémy těchto nástrojů, kterým je třeba se při vlastní implementaci vyvarovat.
Úvod páté kapitoly definuje požadavky, které bude nutné v implementaci splnit. Hlavní částí této kapitoly je návrh architektury a bezpečnostních opatření pro výslednou aplikaci.
Téma poslední šesté kapitoly je samotná implementace navrhnutého nástroje, popis instalace, nastavení a spuštění aplikace. Po kapitolách následuje sturčný
závěr, kde shrnuji zda bylo dostaženo stanovený cílů a uvadím možnosti rozšíření výsledné aplikace.

Nutnou součástí této práce jsou zdrojové kódy aplikace sloužící pro správu souborového systému ZFS dostupné na přiloženém médiu.



